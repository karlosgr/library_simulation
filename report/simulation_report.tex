\documentclass{article}
% \usepackage{listings}
% \usepackage{xcolor}


% \definecolor{atomgray}{RGB}{30, 30, 30}
% \definecolor{atomcyan}{RGB}{86, 156, 214}
% \definecolor{stringatomcolor}{RGB}{206, 145, 120}
% \definecolor{atomblue}{RGB}{28, 103, 190}
% \definecolor{atomyellow}{RGB}{255, 215, 0}
% \definecolor{atompink}{RGB}{186, 134, 192}

% \lstset{
%     language=Python,
%     morekeywords = {True,not,False},
%     morekeywords={while,if},
%     keywordstyle=\color{atompink},
%     classoffset = 0,
%     stringstyle=\color{stringatomcolor},
%     identifierstyle=\color{atomcyan},
%     basicstyle=\small\color{atomgray},
%     morecomment=[l][\color{atomyellow}]{\(}{\)},
%     numbers=left,                   
%     numberstyle=\small\color{black},  
%     frame=single,                    
%     breaklines=true,                 
%     title=\lstname,
%     showlines = false,
% }




% \title{Informe del proyecto \\ Simulaci\'on de eventos discretos}
% \author{Ernesto Roussell Zurita \\Carlos Mario Chang Jard\'{\i}nez\\ Carlos Manuel Garc\'{\i}a Rodr\'{\i}guez \\ C312}
% \begin{document}

% \maketitle
% \begin{abstract}
%     Este es un proyecto totalmente ac\'ademico en el que se busca simular el comportamiento de un sistema de colas en una biblioteca,
%     con el fin de analizar los datos resultantes. 
% \end{abstract}

% \newpage

% \section{Introducci\'on}
%     \subsection*{Descripci\'on del problema}
%     El problema planteado ped\'{\i}a obtener unas estad\'{\i}sticas de un sistema de colas en una biblioteca,
%     dado los datos de la distribuci\'on que segu\'{\i}a la llegada de personas a la biblioteca y el tiempo que
%     se demoraba en atenderlos el bibliotecario deb\'{\i}amos obtener entre otros datos:
%     \begin{itemize}
%         \item El tiempo promedio que una persona pasaba esperando en la cola.
%         \item La longitud promedio de la cola.
%         \item La probabilidad de que se formara cola en la biblioteca.
%         \item Que tanto tiempo libre ten\'{\i} el bibliotecario
%     \end{itemize} 

%     \subsection*{Acercamiento al problema}
%     Nuestra idea de proyecto fue implementar un modelo para simular el transcurso del tiempo
%     en la biblioteca, la llegada de las personas en un tiempo determinado y un bibliotecario 
%     atendi\'endolas, las variables que describen el problema y que hay que tener en cuenta son:
%     \begin{itemize}
%         \item La media de personas llegadas por hora, que sigue una distribuci\'on de Poisson.
%         \item La media de tiempo que se demora el bibliotecario en atender a una persona que sigue una distribuci\'on exponencial.
%     \end{itemize}   

    
%     \section{Detalles de implementaci\'on}
%     Primero deb\'{\i}amos tener una forma de generar variables aleatorias que siguieran las distribuciones Poisson y exponencial. Para ello usamos 
%     la bibloteca $numpy$ de python, que con $numpy.random.poisson$ y $numpy.random.exponential$ genera variables aleatorias siguiendo esas 
%     distribuciones. Ese fue uno de los motivos por el cual elegimos Python como lenguaje para la implementaci\'on del modelo.

%     \subsection*{Abstracci\'on del problema}
%     Para la implementaci\'on definimos los dos tipos de eventos que vamos a manejar en la simulaci\'on: 
%     \begin{itemize}
%         \item Una persona llega a la biblioteca  
%         \item Una persona se termina de atender en el mostrador.
%     \end{itemize}
%     La idea principal es mantener una cola de prioridad de eventos en la cual la prioridad es el tiempo o momento en que ocurrir\'a el evento.
%     En esta cola de prioridad los eventos de llegadas de personas ya vienen determinados, y los eventos de terminar de atender a alguien son los 
%     que se van añadiendo. Gracias a esto estamos logrando simular el tiempo que es una variable continua de forma discreta, ya que el algoritmo 
%     solo va manejando los eventos, y a partir del tiempo que ocurrieron esos eventos es que se recopilan los datos. Para mayor precisi\'on la unidad 
%     de medida del tiempo manejada en la simulaci\'on es segundos.

%     En cada iteracio\'n de la simulaci\'on se almacenaban auxili\'andonos de variables y estructuras de datos lo suguiente: cuanto tiempo estuvo libre 
%     el bibliotecario, cuanto tiempo hubo $i$ personas en la cola, cuanto tiempo esper\'o cada persona en la cola antes de ser atendido, etc. A continuaci\'on
%     mostramos un pequeño fragmento en pseudo-c\'odigo de la parte central del algrotimo que maneja los eventos:

%     \begin{lstlisting}
% while not events.empty():
%     event = events.get()
%     transcur_time = event.time - actual_time
%     actual_time = event.time
%     if free_worker:
%         total_articles += (transcur_time * 22) // 3600
%     number_waiting_persons_in_time[waiting_person_count] += transcur_time
%     if event[1] is EventType.ARRIVE:
%         waiting_persons.put((event.time))
%     if event[1] is EventType.FINISH:
%         free_worker = True
%     if free_worker:
%         if not waiting_persons.empty():
%             person = waiting_persons.get()
%             people_awaiting_time.append(actual_time - person)
%             events.put(
%                 (
%                     np.random.exponential(self._service_delay) + actual_time,
%                     EventType.FINISH,
%                 )
%             )
%             free_worker = False
%     \end{lstlisting}

%     \newpage

%     \section{Resultados}    
% \end{document}

% \documentclass{article}
\usepackage{listings}
\usepackage{xcolor}
% \usepackage{graphicx}
% \usepackage{amsmath}
% \usepackage{caption}

\definecolor{atomgray}{RGB}{30, 30, 30}
\definecolor{atomcyan}{RGB}{86, 156, 214}
\definecolor{stringatomcolor}{RGB}{206, 145, 120}
\definecolor{atomblue}{RGB}{28, 103, 190}
\definecolor{atomyellow}{RGB}{255, 215, 0}
\definecolor{atompink}{RGB}{186, 134, 192}

\lstset{
    language=Python,
    morekeywords = {True,not,False},
    morekeywords={while,if},
    keywordstyle=\color{atompink},
    classoffset = 0,
    stringstyle=\color{stringatomcolor},
    identifierstyle=\color{atomcyan},
    basicstyle=\small\color{atomgray},
    morecomment=[l][\color{atomyellow}]{\(}{\)},
    numbers=left,                   
    numberstyle=\small\color{black},  
    frame=single,                    
    breaklines=true,                 
    title=\lstname,
    showlines = false,
}

\title{Informe del proyecto \\ Simulaci\'on de eventos discretos}
\author{Ernesto Roussell Zurita \\Carlos Mario Chang Jard\'{\i}nez\\ Carlos Manuel Garc\'{\i}a Rodr\'{\i}guez \\ C312}
\begin{document}

\maketitle
\begin{abstract}
    Este es un proyecto totalmente ac\'ademico en el que se busca simular el comportamiento de un sistema de colas en una biblioteca,
    con el fin de analizar los datos resultantes. 
\end{abstract}

\newpage

\section{Introducci\'on}
    \subsection*{Descripci\'on del problema}
    El problema planteado ped\'{\i}a obtener unas estad\'{\i}sticas de un sistema de colas en una biblioteca,
    dado los datos de la distribuci\'on que segu\'{\i}a la llegada de personas a la biblioteca y el tiempo que
    se demoraba en atenderlos el bibliotecario deb\'{\i}amos obtener entre otros datos:
    \begin{itemize}
        \item El tiempo promedio que una persona pasaba esperando en la cola.
        \item La longitud promedio de la cola.
        \item La probabilidad de que se formara cola en la biblioteca.
        \item Que tanto tiempo libre ten\'{\i}a el bibliotecario.
    \end{itemize} 

    \subsection*{Acercamiento al problema}
    Nuestra idea de proyecto fue implementar un modelo para simular el transcurso del tiempo
    en la biblioteca, la llegada de las personas en un tiempo determinado y un bibliotecario 
    atendi\'endolas, las variables que describen el problema y que hay que tener en cuenta son:
    \begin{itemize}
        \item La media de personas llegadas por hora, que sigue una distribuci\'on de Poisson.
        \item La media de tiempo que se demora el bibliotecario en atender a una persona que sigue una distribuci\'on exponencial.
    \end{itemize}   

\section{Detalles de implementaci\'on}
    Primero deb\'{\i}amos tener una forma de generar variables aleatorias que siguieran las distribuciones Poisson y exponencial. Para ello usamos 
    la biblioteca $numpy$ de python, que con $numpy.random.poisson$ y $numpy.random.exponential$ genera variables aleatorias siguiendo esas 
    distribuciones. Ese fue uno de los motivos por el cual elegimos Python como lenguaje para la implementaci\'on del modelo.

    \subsection*{Abstracci\'on del problema}
    Para la implementaci\'on definimos los dos tipos de eventos que vamos a manejar en la simulaci\'on: 
    \begin{itemize}
        \item Una persona llega a la biblioteca.
        \item Una persona se termina de atender en el mostrador.
    \end{itemize}
    La idea principal es mantener una cola de prioridad de eventos en la cual la prioridad es el tiempo o momento en que ocurrir\'a el evento.
    En esta cola de prioridad los eventos de llegadas de personas ya vienen determinados, y los eventos de terminar de atender a alguien son los 
    que se van añadiendo. Gracias a esto estamos logrando simular el tiempo que es una variable continua de forma discreta, ya que el algoritmo 
    solo va manejando los eventos, y a partir del tiempo que ocurrieron esos eventos es que se recopilan los datos. Para mayor precisi\'on la unidad 
    de medida del tiempo manejada en la simulaci\'on es segundos.

    En cada iteraci\'on de la simulaci\'on se almacenaban auxili\'andonos de variables y estructuras de datos lo siguiente: cuanto tiempo estuvo libre 
    el bibliotecario, cuanto tiempo hubo $i$ personas en la cola, cuanto tiempo esper\'o cada persona en la cola antes de ser atendido, etc. A continuaci\'on
    mostramos un pequeño fragmento en pseudo-c\'odigo de la parte central del algoritmo que maneja los eventos:

    \begin{lstlisting}
while not events.empty():
    event = events.get()
    transcur_time = event.time - actual_time
    actual_time = event.time
    if free_worker:
        total_articles += (transcur_time * 22) // 3600
    number_waiting_persons_in_time[waiting_person_count] += transcur_time
    if event[1] is EventType.ARRIVE:
        waiting_persons.put((event.time))
    if event[1] is EventType.FINISH:
        free_worker = True
    if free_worker:
        if not waiting_persons.empty():
            person = waiting_persons.get()
            people_awaiting_time.append(actual_time - person)
            events.put(
                (
                    np.random.exponential(self._service_delay) + actual_time,
                    EventType.FINISH,
                )
            )
            free_worker = False
    \end{lstlisting}

\newpage

\section{Resultados}
    A continuación se muestran los resultados obtenidos tras realizar la simulación durante 100 días. Los resultados principales incluyen:
    \begin{itemize}
        \item El tiempo promedio que una persona pasa esperando en la cola.
        \item La longitud promedio de la cola.
        \item La cantidad total de artículos procesados por el bibliotecario.
    \end{itemize}

    \begin{itemize}
        \item Tiempo promedio de espera: de 3 a 4 minutos.
        \item Longitud promedio de la cola: 0.42 personas.
        \item Total de artículos procesados por día: 44
    \end{itemize}
\section{Conclusiones}
    A partir de la simulación, podemos concluir que:
    \begin{itemize}
        \item El tiempo de espera promedio en la cola es razonablemente bajo, lo cual sugiere que el sistema de atención es eficiente.
        \item La longitud promedio de la cola indica que raramente se forman colas largas, lo cual es deseable en un entorno de biblioteca.
        \item El bibliotecario tiene un tiempo libre moderado, permitiéndole realizar otras tareas necesarias.
    \end{itemize}

    La simulación proporciona una herramienta útil para analizar y mejorar el sistema de atención en la biblioteca. Futuras mejoras al modelo podrían incluir 
    variaciones en los patrones de llegada y servicio, así como la inclusión de múltiples bibliotecarios para analizar su impacto en el rendimiento del sistema.

\end{document}
